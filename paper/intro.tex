

The exceeding growth of big datasets has created a need for using distributed systems and 
accelerators for real-time analytics, including for graph and social network analytics. Social 
networks such Facebook and Twitter are constantly growing  at time of writing have over #### 
members with **** relationships between these members.
The decrease of transistor sizes and the improved power consumption, as part of Moore's laws, has 
brought a new challenge with it - an increased number of faults due to random bit flipping. This 
problem is very likely to become even more challenging as transistor sizes continue to decrease and 
the power envelope restrictions become even tighter. As larger systems, with a higher thread count, 
become ubiquitous so will the problem of power related faults.

In this work, we tackle the problem of faults in a connected component algorithm that is 
propagation based. We modify the parallel algorithm of Shiloach-Vishkin \cite{shiloachvishkin} and 
make it fault-tolerant by augmenting the data structures used by the algorithm and by adding a 
detection and correction scheme at the end of every iteration. By detecting and correcting the 
error at the end of each iteration we are able to limit the propagation of the error to the 
remainder of the graph in the following iterations.

Main contributions:
* Little overhead when there aren't any faults.
* Same number of iterations as a fault-free execution for reasonable error rates.
* Overhead if rather constant in time until the number of threads is significantly high.
* 



