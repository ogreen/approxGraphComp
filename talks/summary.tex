\lyxframe{Summary of Contributions }

\begin{alertblock}{Self-correcting Algorithms}

We introduce a new fault tolerant algorithm design principle that we call \emph{self-correction}. A self-correcting algorithm remains in a valid state, despite the faulty execution of an iteration, under the assumption that its previous state was a valid one. 
\end{alertblock}
\pause
\begin{exampleblock}{Self-Correcting Connected Components Algorithm}
\begin{itemize}
\item We apply the ideas of self-correction to Label-propagation algorithm for graph connected component problem.
\item Assumes availability of selective reliability mode.
\item Requires $\bigO{V}$ additional storage and computations per iteration compared to $\bigO{|V|+|E|}$ cost for the baseline algorithm.
\item 10-35\%  increases in execution time for one error for 64 memory operations.
\end{itemize}
\end{exampleblock}

\lyxframeend{}