We term fault as any instance of hardware deviating from its expected
behiour. Impact of such faults on the application can vary significantly
depending on location of the fault. Based on impact of the faults,
they can be classified into two braod categories: hard fault and soft
faults. A hard fault causes the application to abort for instance
failing of nodes and network fall into these categories. On the other
hand, soft faults may not cause application to abort. Nevertheless,
a soft fault can potentially lead to an incorrect result, which we
term as failure. Bitflips in memory and latches are example of softfaults.
A more detailed explaination of types of faults can be found elsewhere
{[}Hommen,heroux{]}.

In this paper, we only deal with soft faults. A particularly insidious
manifestation of soft-faults is silent data corruption. Silent data
corruption occurs when a soft fault leads to corruption of entire
intermidiate variables, without notifying the application. In such
cases, application may arrive at an incorrect solution and yet, inform
user as correct solution. Thus, silent data corruption can lead serous
reliablity issues in the computing. 

The importance of having algorithmic level fault tolerance has already
been explored for a number of numerical computations{[}citation{]}.
Graph computations are equally susceptible to softfautls as numerical
computation. Researchers have started looking at resilient discrete
computation only recently {[}cite{]}
