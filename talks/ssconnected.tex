\lyxframe{Self-stabilizing Connected-component Algorithm- Problem Statement}

\begin{itemize}
\item Given an arbitrary (valid or invalid) state $S={CC,P}$, determine if it is a valid state? 
\item If invalid then construct a valid state $S^{*}$ close to $S={CC,P}$.
\end{itemize}

\lyxframeend{}




\lyxframe{ Parent Tree}

\begin{itemize}
\item Parent Graph $\mathcal{H}$: Subgraph  of $G$ that consists of only edges $(v,P[v])$
\end{itemize}

\lyxframeend{}


\lyxframe{Parent forest}

\begin{itemize}
\item In a fault free execution of label propagation algorithm,  $\mathcal{H}$ is a set of disjoint trees, or \emph{forest}.
\end{itemize}

\lyxframeend{}



\lyxframe{Parent Tree}

\begin{itemize}
\item In a fault free execution, $CC[v]\geq CC[P[v]]$ 
\item If $v$ is a root of a tree, then  $CC[v] = v$ 
\end{itemize}

\lyxframeend{}


\lyxframe{Valid states}

\begin{theorem}

A label propagation state $S=(CC,P)$ is a valid state---i.e.,
a fault-free execution of algorithm  starting from $S$ will converge to the correct solution---if, for all vertices $v$, 

\begin{itemize}
\item $P[v] \in \mathcal{N}(v)$
\item The graph $H=(V, \{\{v,P(v)\}\forall v \in V\} $ does not contain any loops other than self-loops
\item $CC[v] \geq CC[P(v)] $
\end{itemize}

\end{theorem}


\lyxframeend{}



\lyxframe{Verifying Valid states}

\begin{itemize}
\item $P[v] \in \mathcal{N}(v)$ can be done in $\mathcal{O}(V)$
\item $CC[v] \geq CC[v] $  can be done in $\mathcal{O}(V)$
\item The graph $H=(V, \{\{v,P(v)\}\forall v \in V\} $ can be done in $\mathcal{O}(V)$ operation
by using Tarzen's strongly connnected component (SCC) algorithm.
\item Tarzen's strongly connnected component (SCC) is not parallel enough to implement in vertex centric model. 
\end{itemize}

\lyxframeend{}


\lyxframe{Parallel Loop Detection in $\mathcal{H}$ }

\begin{itemize}
\item Calculate $K(v) = \min \{P(v),P{^2}(v), P{^3}(v) \ldots \} $
\end{itemize}

\begin{theorem}
\begin{itemize}
\item If there is no loop, then $K(v) < v$ for all $v$ except for the root of a tree. 
\item If $K(v)=v$, then there is a loop, and $v$ has minimum vertex id among all the element in the loop. 
\end{itemize}
\end{theorem}

\lyxframeend{}


\lyxframe{Calculating $K(v) = \min \{P(v),P{^2}(v), P{^3}(v) \ldots \}$ }

\begin{itemize}
\item Calculating  $K(v)$ is a \emph{prefix} sum operation.
\item Initialize: $K(v) = P(v)$
\item Update: $K(v) = \min \{ K(v), P(K(v)) \}$
\item Terminate when no $K(v)$ changes. 
\end{itemize}

\begin{itemize}
\item Cost $\mathcal{O}(V \log V)$
\end{itemize}

\lyxframeend{}

\lyxframe{ Verifying Valid  of a State}

\begin{itemize}
\item $P[v] \in \mathcal{N}(v)$
\item The graph $H=(V, \{\{v,P(v)\}\forall v \in V\} $ does not contain any loops other than self-loops
\item $CC[v] \geq CC[P(v)] $ (Not needed)
\end{itemize}


\lyxframeend{}


\lyxframe{ State Correction}
Hyatt Place Albuquer Uptown
\begin{itemize}
\item $P[v] \notin \mathcal{N}(v)$, reset $v$, $CC[v]=v$, $P(v)=v$
\item Calculate $K(v)$. If $K(v)=v$, break the loop at $v$, $P(v)=v$.
\item Set $CC[v] = K(v)$.
\end{itemize}


\begin{itemize}
\item Previosuly used periodic correction scheme to bring it to a valid state.
\item Often Label propagation algorithm takes very few iteration to converge.
\item We perform correction step only once at the end of label propagation.
\end{itemize}

\lyxframeend{}


\lyxframe{ A Practical Fault Tolerant Label-propagation algorithm}

\begin{itemize}
\item Frequently 2-loops are formed. 
\item Slows down the convergence.
\end{itemize}


\begin{itemize}
\item In every iteration, we perform some checks to stop fault propagation.
\begin{enumerate}
	\item $P(v) \in \mathcal{N}(v)$
	\item $CC[v] \geq CC[p(v)]$
	\item Two loop detection: $v = P(P(v))$.
\end{enumerate} 


\item Cost $\mathcal{O}(V)$ 
\item These checks can be unreliable.
\end{itemize}

\lyxframeend{}


\lyxframe{ To summerize }

\begin{itemize}
\item Requires additional O(v) memory
\item Costs log(v)
\item May take additional iterations.
\end{itemize}

\lyxframeend{}

