%% LyX 2.2.2 created this file.  For more info, see http://www.lyx.org/.
%% Do not edit unless you really know what you are doing.
\documentclass[conference, english]{IEEEtran}
\usepackage{fouriernc,amsmath}
\usepackage[T1]{fontenc}
\usepackage[latin9]{inputenc}
\usepackage{color}
\definecolor{note_fontcolor}{rgb}{0.800781, 0.800781, 0.800781}
\usepackage{verbatim}
\usepackage{float}
\usepackage{amsmath}
\usepackage{amsthm}
\usepackage{graphicx}

\makeatletter

%%%%%%%%%%%%%%%%%%%%%%%%%%%%%% LyX specific LaTeX commands.
%% The greyedout annotation environment
\newenvironment{lyxgreyedout}
  {\textcolor{note_fontcolor}\bgroup\ignorespaces}
  {\ignorespacesafterend\egroup}
%% A simple dot to overcome graphicx limitations
\newcommand{\lyxdot}{.}

\floatstyle{ruled}
\newfloat{algorithm}{tbp}{loa}
\providecommand{\algorithmname}{Algorithm}
\floatname{algorithm}{\protect\algorithmname}

%%%%%%%%%%%%%%%%%%%%%%%%%%%%%% Textclass specific LaTeX commands.
\theoremstyle{plain}
% \newtheorem{thm}{\protect\theoremname}

%%%%%%%%%%%%%%%%%%%%%%%%%%%%%% User specified LaTeX commands.
% \documentclass[english]{sig-alternate-05-2015}
\usepackage{calc}
%\usepackage{amsthm}
\usepackage{verbatim}


%% without following commands, amsthm complains proof being already defined 
\let\proof\relax
\let\endproof\relax
\usepackage{amsthm}
\usepackage{graphicx}


\usepackage{mdwlist}% compact lists

\usepackage{relsize}% relative font sizes
\usepackage{todonotes}% private notes
\usepackage{breqn}


\usepackage{soul}


\usepackage{color}


% \usepackage{hyperref}
\PassOptionsToPackage{pdfpagelabels=false}{hyperref} 
% \hypersetup{bookmarks=true,bookmarksnumbered=true,pdfstartview=FitB,colorlinks=true,pdfborder=0 0 0}



% \usepackage{xspace}

% \usepackage[sort&compress]{natbib}

\usepackage{comment}




% Test edit via git by Rich

%% for manually breaking cells in tables.
\newcommand{\specialcell}[2][c]{%
  \begin{tabular}[#1]{@{}l@{}}#2\end{tabular}}


%%%%%%%%%%%%%%%%%%%%%%%%%%%%%% LyX specific LaTeX commands.
%% Because html converters don't know tabularnewline
\providecommand{\tabularnewline}{\\}


%%%%
\usepackage{booktabs}
\newcommand{\ra}[1]{\renewcommand{\arraystretch}{#1}}


%%%%%%%%%%%%%%%%%%%%%%%%%%%%%% Textclass specific LaTeX commands.
\theoremstyle{plain}
\newtheorem{thm}{\protect\theoremname}\theoremstyle{definition}
\newtheorem{defn}{\protect\definitionname}

%%%%%%%%%%%%%%%%%%%%%%%%%%%%%% User specified LaTeX commands.
\usepackage{algorithm}



% \usepackage{algpseudocode}
\usepackage[noend]{algpseudocode}%to remove endif, and endfors from pseudocode




\usepackage{babel}
\providecommand{\definitionname}{Definition}
\providecommand{\theoremname}{Theorem}
\usepackage{balance}

%========== Custom typographic commands ==========
\newcommand{\emailAddr}[1]{\texttt{\smaller {#1}}}
\newcommand{\mailtoLink}[2]{\href{mailto:#2}{\emailAddr{#1}}}
\newcommand{\refSection}[1]{Section~\ref{#1}}
\newcommand{\refsec}[1]{\S~\ref{#1}}
\newcommand{\refequ}[1]{eq.~\eqref{#1}}
\newcommand{\refEquation}[1]{Equation~\eqref{#1}}
\newcommand{\refFigure}[1]{Figure~\ref{#1}}
\newcommand{\reffig}[1]{fig.~\ref{#1}}
\newcommand{\refDefinition}[1]{Definition~\ref{#1}}
\newcommand{\refdef}[1]{defn.~\ref{#1}}
\newcommand{\refAlgorithm}[1]{Algorithm~\ref{#1}}
\newcommand{\refalg}[1]{alg.~\ref{#1}}
\newcommand{\refAlg}[1]{Alg.~\ref{#1}}
\newcommand{\refTable}[1]{Table~\ref{#1}}
\newcommand{\reftab}[1]{table~\ref{#1}}
\newcommand{\refTheorem}[1]{Theorem~\ref{#1}}
\newcommand{\refthm}[1]{thm.~\ref{#1}}
\newcommand{\refprep}[1]{prepostion~\ref{#1}}
\newcommand{\ramki}[1]{{\color{red}{\bf [Ramki: #1]}}}

%========== Document Specific commands ==========

\setlength{\textfloatsep}{0pt}% Remove \textfloatsep
\newcommand{\etal}{\textit{et al}.}

\newcommand{\sv}{LP}
\newcommand{\ftsv}{\textsc{FT-LP}}
\newcommand{\graphname}[1]{\emph{#1}}

\newcommand{\CCVAL}{$CC$}
\newcommand{\icomment}[1]{\textbf{ \textcolor{red} {#1}}}
\newtheorem{prop}{Proposition}\clubpenalty=10000 
\widowpenalty = 10000

\@ifundefined{showcaptionsetup}{}{%
 \PassOptionsToPackage{caption=false}{subfig}}
\usepackage{subfig}
\makeatother


\begin{document}


\author{
\IEEEauthorblockN{Piyush Sao}%
\IEEEauthorblockA{
Georgia Institute of Technology\\
email: piyush3@gatech.edu \\
}  \and
\IEEEauthorblockN{Richard Vuduc}%
\IEEEauthorblockA{
Georgia Institute of technology\\
email: richie@gatech.edu \\
}
}

\title{A Self-Stabilizing Connected Components Algorithm }
\maketitle

\section{Introduction}
\label{sec:intro}


The exceeding growth of big datasets has created a need for using distributed systems and 
accelerators for real-time analytics, including for graph and social network analytics. Social 
networks such Facebook and Twitter are constantly growing  at time of writing have over #### 
members with **** relationships between these members.
The decrease of transistor sizes and the improved power consumption, as part of Moore's laws, has 
brought a new challenge with it - an increased number of faults due to random bit flipping. This 
problem is very likely to become even more challenging as transistor sizes continue to decrease and 
the power envelope restrictions become even tighter. As larger systems, with a higher thread count, 
become ubiquitous so will the problem of power related faults.

In this work, we tackle the problem of faults in a connected component algorithm that is 
propagation based. We modify the parallel algorithm of Shiloach-Vishkin \cite{shiloachvishkin} and 
make it fault-tolerant by augmenting the data structures used by the algorithm and by adding a 
detection and correction scheme at the end of every iteration. By detecting and correcting the 
error at the end of each iteration we are able to limit the propagation of the error to the 
remainder of the graph in the following iterations.

Main contributions:
* Little overhead when there aren't any faults.
* Same number of iterations as a fault-free execution for reasonable error rates.
* Overhead if rather constant in time until the number of threads is significantly high.
* 





\section{Connected components}
\label{sec:connected}

\paragraph{Label Propagation algorithm in Fault Free execution}

\begin{figure*}

\includegraphics[width=0.2\paperwidth]{figure/sv1}\includegraphics[width=0.2\paperwidth]{figure/sv2}\includegraphics[width=0.2\paperwidth]{figure/sv3}\includegraphics[width=0.2\paperwidth]{figure/sv4}\includegraphics[width=0.2\paperwidth]{figure/sv5}\caption{These sub-figures conceptually show how the connected component id
propagates through the graph as time evolves. Subfigures represent
snapshots of the algorithm at different times. For simplicity, this
example assumes that all the shown vertices are connected. Initially
the number of connected components is equal to the number vertices.
(a) Depicts the initial state in which each vertex is in its own component.
(b)-(d) depict that some vertices belong to the same connected component
yet may require multiple label updates (in either the same iteration
or a separate iteration). (e) is the final state in which there is
a single connected component.}

\end{figure*}

\begin{table}

\centering
\footnotesize
\caption{Symbols used in the fault free \sv algorithm and in the new fault tolerant algorithm.}
\ra{1.2}
\begin{tabular}[t]{@{}lll@{}}
\toprule
Symbol 					& Decription 						& Size\\
\midrule
$V$ 					& Vertices in the graph 			& O(V)\\
$E$ 					& Edges in the graph 				& O(E)\\
$adj(v)$ 				& Adjacency list for vertex $v\in V$& \\
\CCVAL 					& Connected component array 		& O(V)\\
\CCVAL$^i$ 				& \specialcell{Connected component array \\after iteration} $i$ & O(V)\\
\CCVAL$^\infty$ 		& \specialcell{The final connected component \\mapping upon algorithm completion}. & O(V)\\

						& Fault free \\

$H$ 					&  & O(V)\\
\bottomrule 
\end{tabular}
\label{tab:symbols}

\end{table}

The aim of label propagation algorithm is to mark every vertex in
a given connected component with a pre-defined common label. This
common label is same within each connected component, and different
across different connected component. There are many choice for the
common label. As a convention, we consider minimum vertex-id in the
connected component as the final common label. 

The label propgation algorithm keeps an array $CC$ of current labels
for all vertices. For each vertex $v$, its label $CC[v]$ is initalized
with its vertex-id $CC[v]=v$. In every iteration, each vertex updates
its label by calculating minimum label of all its neighbours and itself:
\begin{equation}
CC^{i+1}[v]=\min_{u\in\mathcal{N}(v)}CC^{i}[u],\label{eq:lp_update_eqn}
\end{equation}

where $\mathcal{N}(v)=\left\{ v,adj(v)\right\} $ is the defined as
immidiate neighbourhood of $v$. Thus, through out the iteration minimum
vertex id propagates to all the vertces in the connect component.
The iteration converges when there are no more label changes in the
graph. 

Depending on the constrints, equation \ref{eq:lp_update_eqn} can
be implmented in two ways. In the first way, we use two different
arrays to store $CC^{i+1}$ and $CC^{i}$. We refer to this implementation
is Sync LP algorithm. In an another way, we overwrite $CC^{i+1}$
on $CC^{i}$. We refer to this version as Async LP algorithm. Depending
on architecture and programming model, the two variants may have different
performance characteristics. In subsequent discussion, we assume Async
LP algorithm. Yet, our results are equally applicable to both instance
of the LP algorithm.

Each iteration of LP, visits all the vertex and edges once and thus
costs $\mathcal{O}(V+E)$. The LP algorithm requires $\mathcal{O}(d)$
iterations to converge, where $d$ is the diameter of the graph. We
may use short-cutting to bound the number of iteration to $\mathcal{O}(log(d))$
{[}insert citation{]}. However, in practice async LP algorithm only
takes a few more iteration than implementing full short cutting step,
without the cost of short cutting. 


\section{Impact of hardware faults on LP algo}
\label{sec:fault}

We term fault as any instance of hardware deviating from its expected
behavior. Impact of such faults on the application can vary  depending on the
location of the fault. Based on the impact of the faults, they can be
classified into two broad categories: hard fault and soft faults. A hard fault
causes the application to terminate prematurely. For instance, failing of nodes and network fall
into this category. On the other hand, soft faults may not cause the
application to abort. Even so, a soft fault can lead to an incorrect result,
which we term as the failure. Bitflips in memory and latches are examples of
soft faults. Interested readers can find a more detailed discussion on faults
elsewhere  \cite{hoemmen2011ftgmres}.

In this paper, we only deal with soft faults. A particularly insidious
manifestation of soft-faults is silent data corruption. Silent data corruption
occurs when a soft fault leads to corruption of entire intermediate variables,
without notifying the application. In such cases, the application may arrive
at an incorrect solution and yet, report it as correct solution. Thus, silent
data corruption can lead to serious reliability issues in the computing. 


The importance of having algorithmic level fault tolerance has already been
explored for a number of numerical computations (see \cref{sec:related}). Graph
computations are equally susceptible to soft faults as numerical computation.
Researchers have started looking at resilient discrete computation only
recently (\TODO{cite})



\section{Self-stabilzing Algorithms}
\label{sec:sc-algm} %describes general ss and SC algorithm

\textbf{\emph{}}%
% \begin{comment}
% \textbf{\emph{I}}\emph{ntro to terminology} 
% \begin{itemize}
% \item states of the algorithm 
% \item valid state and in valid state 
% \item effect of fault on valid state 
% \end{itemize}
% \end{comment}

Formally, a system is said to be self-stabilizing, if starting from
any arbitrary \textquotedblleft state\textquotedblright , it comes
to a valid state in a finite number of steps{[}dijkstra{]}. An algorithm
can also be viewed as a system with states and transitions. Its state
is a subset of the intermediate variable which enables continued execution.
A state of the algorithm is said to be in a valid state if the algorithm
will converge to a correct solution in fault-free execution starting
from this state, otherwise invalid. In previous work{[}{]}, we have
shown that such abstraction may help us to construct resilient algorithms. 

% \begin{comment}
% \textbf{\emph{Self-stabilization in Algorithmic resilience}}
% \begin{itemize}
% \item define 
% \item give examples 
% \end{itemize}
% \end{comment}


\paragraph{Impact of hardware fault on algorithms state}

If an algorithm is well suited for a given problem then in a fault-free
execution, its state remains valid during entire computation. However,
soft-fault such as bit flip can corrupt the intermediate variables
and, can potentially bring it to an invalid state. Subsequent fault-free
execution will lead to an incorrect solution, thus failure, unless
it is brought to a valid state by some mechanism. Our principle for
designing fault-tolerant algorithm is based on the idea of augmenting
the algorithm so that it can bring itself to a valid state, by design.
We distinguish between following two principles for bringing the algorithm
to a valid state.

\paragraph{Self-correcting algorithm}

A self-correcting algorithm can bring itself to a valid state by correcting
its state with information of a previous valid state. In contrast
to self-stabilization algorithm, a self-correcting must start from
a valid state. In reality, it is not such a limitation as most algorithms
by design starts from a valid state. On the other hand, the self-correcting
algorithm can be more efficient than the self-stabilizing algorithm,
as it can meaningfully exploit information of a previous valid state.
In {[}cite{]}, we presented a self-correcting of SYNC version of label
propagation algorithm. 

% \begin{comment}
% \textbf{\emph{s}}\emph{elf-stabilizing algorithm} 
% \begin{itemize}
% \item define 
% \item give examples 
% \end{itemize}
% \textbf{\emph{s}}\emph{elf-correcting algorithm} 
% \begin{itemize}
% \item define 
% \item give examples 
% \end{itemize}
% \textbf{\emph{A}}\emph{ comparison between SS and SC} 
% \begin{itemize}
% \item relation between SS and SC (SS is special case of SC) 
% \item advantage of SS over SC 
% \end{itemize}
% \end{comment}



\section{Self-correcting Label Propagation Algorithm}
\label{sec:ft-connected} 
We answer the two related questions to construct a self-stabilizing label-
propagation algorithm. First, how do we determine if a given arbitrary state
$S={CC, P}$ of \cref{alg:SV_ALG} is valid? Secondly,  how do we construct a
provable valid $S^{*}$, which in some measure, is close to the given state $S$?

First, we prove the following set conditions $S$ is sufficient to ensure its
validity.  
\begin{thm}
\label{thm:ss_valid}
Starting from state $S= \{CC,P \}$, the \cref{alg:SV_ALG} will converge to correct solution for graph $G=\{ V, E\}$, if $S$ satisfies the following conditions.
\begin{enumerate}
\item $CC[v]\leq v$  for all $v\in V$;
\item $P(v)\in\mathcal{N}(v)$ for all $v\in V$;
\item $CC[P(v)]\leq CC[v]$ for all $v\in V$; and 
\item Directed graph described by parent array $H=(V,E_{H})$, where $E_{H}=\left\{ (v,P(v)),\forall v\in V\right\} $
describes a forest. 
\end{enumerate}
\end{thm}
Using \cref{thm:ss_valid}, we present a  correction step in \TODO{Crosref to correction step}, which brings the
\cref{alg:SV_ALG} from an arbitrary state $S$ to a valid state $S^{*}$ in a
computationally efficient way.


% In this section, we will describe the design of the self-stabilizing
% label propagation algorithm. We start with decribing the state of
% the algorithm and a condition for a valid state. However, the condition
% for valid state is not easy to verify computationally. To overcome
% this issue, in we augmented the data structure with redundancy to
% allow state verification. We use the augmented state as previously,
% and present new conditions for valid states. And finally we give efficient
% algorithm to verify these algorithms in cost effective manner.



\subsubsection*{State of LP algorithm}

In algo{[}REF{]}, the current connected component label array $CC$
describes the state of LP algorithm. We use $CC^{i}[v]$ to denote
the current label for vertex $v$ in $i$-th iteration. We denote
the final label of vertex $v$ in fault-free execution as $CC^{\infty}[v]$. 

\subsubsection*{Effect of fault in LP state}

\subsubsection*{Valid states}
\begin{thm}
A connected component array $CC$ is a valid state-{}-{}-i.e., a fault-free
execution of algorithm (CITE) starting from $CC$ will converge to
the correct solution-{}-{}-if, for all vertices $v$, $CC^{\infty}[v]\leq CC[v]\leq v$.
\end{thm}

\subsection{\emph{Self-correcting Label Propagation Algorithm}}

We described self-correcting label propagation algorithm for synchronoous
label propagation algorithm. Self-correcting algorithm corrects the
current state using previous valid state. 

To do so, we show that if $CC^{i-1}$ is valid state then $CC^{i}$
is valid if 
\begin{equation}
CC^{i}[v]=CC^{i-1}[u]\ where\ u\in\mathcal{N}(v)\ \forall v
\end{equation}

However, verifying eq{[}x{]} still requires $O(V+E)$ work, thus inefficient.
However, self-correcting algorithm uses an auxilliary data structure
called parent array. Parent of vertex $v$, denoted by $P[v]$ , refers
to vertex that 
% \begin{itemize}
% \item introduce parent array 
% \item what do you check and how do you check 
% \item Limitations 
% \begin{itemize}
% \item fault propagation 
% \item is not applicable to asynchronous version 
% \end{itemize}
% \end{itemize}

\subsection{\emph{Self-stabilzing Label Propagation Algorithm} }

The main difference between self-correcting label propagation algorithm
and self-stabilizing label propagation algorithm is that self-stabilizing
label propagation algorithm doesn't require a previous valid state
$S^{i-1}$ to bring the algorithm to a valid state. We can formally
pose this question as follows.

\paragraph*{Problem Statement:}

Given a graph $G=\left(V,E\right)$ and an arbitrary state for label
propagation $S=\left(CC,P\right)$, determine if $S$ is a valid state.
If $S$ is invalid, then construct a state $S^{*}$such that $S^{*}\approx S$,
and $S^{*}$ is a valid state. 

\subsubsection{Properties of LP state in fault free execution}

We describe two key properties of a state $S$ in fault free execution
and later we prove that for any state that satisfy these properties
is a valid state. 
\begin{enumerate}
\item $CC[v]\leq v$;
\item $P(v)\in\mathcal{N}(v)$;
\item $CC[P(v)]\leq CC[v]$; and 
\item Directed graph described by parent array $H=(V,E_{H})$, where $E_{H}=\left\{ (v,P(v)),\forall v\in V\right\} $
describes a forest. 
\end{enumerate}
Self-correcting label propagation algorithm uses property 1 and 2,
and additionally assumes that $CC^{i-1}$ is a valid state. 

Property 3 describes the range of valid values of $CC[v]$. In a synchronous
label propagation algorithm, we have $CC^{i}[v]=CC^{i-1}[P(v)]$,
and since label decreases over the iteration, thus latest $CC^{i}[P(v)]\leq CC^{i-1}[P(v)]$.
Therefore, $CC^{i}[P(v)]\leq CC^{i}[v]$. While we do not show here
in detail, this relation holds good in fault free execution of asynchronous
case also. 

Property 4 describes the structure of parent array. Make a gifure
and explain. 

\subsubsection{Sufficient condtions for a valid LP state }

We show that the former four conditions are sufficient for a valid
state. 
\begin{thm}
A label propagation state $S=\left\{ CC,P\right\} $ is a valid state
if for all vertices $v$, we have following conditions met:
\end{thm}
\begin{enumerate}
\item $CC[v]\leq v$;
\item $P(v)\in\mathcal{N}(v)$;
\item $CC[P(v)]\leq CC[v]$; 
\item If $P(v)=v$, then $CC[v]=v$; and 
\item Directed graph described by parent array $H=(V,E_{H})$, where $E_{H}=\left\{ (v,P(v)),\forall v\in V\right\} $
describes a forest.
\end{enumerate}
\begin{proof}
If $H=\left(V,E_{H}\right)$ is a forest then, for any vertex $v$,
the sequence $\left\{ P(v),P^{2}(v),\ldots\right\} $ convereges to
the root of the tree in the forest $H$. Let's denote the convergent
of the sequence as $P^{\infty}(v)$. 

Let $u=P^{\infty}(v$). Since $P(u)=P(P^{\infty}(v))=P^{\infty}(v)=u$,
therefore from condition 4:
\begin{equation}
CC[u]=u
\end{equation}

If condition-2 holds for all the vertices $\left\{ v,P(v),P^{2}(v),\ldots\right\} $,
then $v$ and $P^{\infty}(v)=u$ are in same connected component.
Therefore,
\begin{equation}
CC^{\infty}[v]=CC^{\infty}[u].
\end{equation}

Since label of $u,$will monotonically decrease in susequent iteration,
therefore 
\begin{equation}
u\ge CC^{\infty}[u]
\end{equation}

From condition-3 $CC[v]\geq CC[P(v)]\geq CC[P^{2}(v)]...$
\begin{equation}
CC[v]\geq CC[P^{\infty}(v)]=CC[u]=u
\end{equation}

Combining eq(x) and eq(y) and eq(z), we get $CC[v]\ge u\geq CC^{\infty}[u]=CC^{\infty}[v]$
\begin{equation}
CC[v]\geq CC^{\infty}[v]
\end{equation}

Combining eq(x) with condition 1, we get 
\[
CC^{\infty}[v]\le CC[v]\le v
\]

Eq(x) holds for all the vertices $v\in V$, therefore by Thm(1), $S=\left\{ CC,P\right\} $
is a valid state.
\end{proof}
%
Note that Thm(2) describes a set of sufficient condition. In general,
there can be a state from which algorithm will converge to correct
solution yet not satisfy Thm(2), leading to false positives. However,
false-positives will only cause a small amount of additional computation,
which is not a big concern in this case. 

\subsection{Verifying state validity}

All the conditions of Thm(2) except the last one, can be verified
locally for any vertex in $\mathcal{O}(1)$ operations. To verify
$H=(V,E_{H})$ is a forest, is equivalent to verifying that $H$ does
not have any loops, which is not a local operation.

\subsubsection{Loop detection in $H$}

In sequential case, we can use Tarzen's strongly connected component
algorithm to find the loops in the graph $H$. Tarzen's strongly connected
component algorithm requires $\mathcal{O}(V+E_{H})$ operations. But
since, $|E_{H}|=|V|$, therefore detecting loop in $H$ requires $\mathcal{O}(V)$
operations in total, which is same as cost of other operations in
sequential case.

In parallel case, Tarzen's strongly connected component algorithm
(or any other algorithm based on BFS or DFS) are not suitable as:
(a) they are inhrently sequential and have a depth of $\mathcal{O}(V)$;
and (b) they can not be efficiently expressed in vertex centric programming
model as label propagation algorithm. Therefore, we need a new loop
detection algorithm which is parallel and can be expressed in vertex
centric programming model. 

\subsubsection{Parallel Loop detection in $H$}

\begin{algorithm}
\begin{algorithmic}[1]
\Require{$\rho \geq 1$}
\Ensure{$X_k$}
\State{$NumChanges \leftarrow 1$}
\State{ $R(v) \leftarrow P(v)$}
\While{$ NumChanges >0$}
	\For{all vertex $v\in V$, \emph{\textbf Do in Parallel} }
		\State{$ r= \min (R(v), R(P(v))) $}
		\If{$r \neq R(v) $}
			\State{$R(v)= v$}
			\State{$NumChanges =NumChanges+1$}
		\EndIf
		\If{$v = R(v) $}
			\State{\emph{\textbf Report Loop} }
		\EndIf
	\EndFor
\EndWhile
\end{algorithmic}\caption{}
\end{algorithm}

Our parallel loop detection is based on the idea that if $v$ is a
part of the loop in $H$, then $v$ will re-appear in the sequence
$\left\{ P(v),P^{2}(v),\ldots\right\} $. Let's call the minimum element
in the sequence $\mathcal{A}(v)$:
\[
\mathcal{A}(v)=\min\left\{ P(v),P^{2}(v),\ldots\right\} 
\]

If there is no loop in the, then $v$ will not appear in the sequence
$\left\{ P(v),P^{2}(v),\ldots\right\} $, thus $v$ can not be equal
to $\mathcal{A}(v)$. Therefore, $v=\mathcal{A}(v)$ is only possible
if there is a loop in $H$ which consist of vertices $\left\{ v,P(v),P^{2}(v),\ldots\right\} $.
Thus to detect any loop in $H$, we calculate $\mathcal{A}(v)$ for
all the vertices and compare it with $v$. 

We calculate $\mathcal{A}(v)$ in parallel using pointer jumping technique.
The Algm{[}x{]} describes the pseudocode for parallel loop detection
in $H$. ADD more text here

\paragraph*{Cost of loop detection}

If there is no loop in $H$, then Algm{[}x{]} requires $\lceil log_{2}(h)\rceil+1$
iterations to converge, where $h$ is the maximum height among all
the trees in the forest $H$. If there are cycles in $H$, and if
length of maximum cycle is $c$then, Algn{[}x{]} takes $\lceil log_{2}(c)\rceil+2$
iterations to detect it. So it need a total of $\max\left(\lceil log_{2}(h)\rceil+1,\lceil log_{2}(c)\rceil+2\right)$
iterations. In each iteration, we perform $\mathcal{O}(V)$ work.
Thus total cost of loop detection is $\mathcal{O}(Vlog(V))$ as $c,\ h\leq|V|$. 

\subsection{Correction step}

Correction step has two parts: detection of invalid component of states;
and bring them to a correct state. Using Algm{[}x{]} and Thm{[}y{]},
we design the correction step as follows. First we ensure that graph
$H$ is a proper subgraph of the input graph $G$, by verifying $P(v)\in\mathcal{N}(v)$
for all vertex $v$. If for some vertex $P(v)\notin\mathcal{N}(v)$,
then we reset vertex $v$ by setting $CC[v]=v$ and $P(v)=v$. After
this we check, if a vertex $v$ is a root of any tree, then $CC[v]=v$.
It should be noted that in both check, if we find that state of $v$
is invalid, it also means that state of all the descendent of $v$
is also invalid. However, all the otehr vertices which have invalid
state have not been informed yet. Once these checks and local correction
are performed, we use a modified algorithm to check for loops and
correct labels. 

\begin{algorithm}
\begin{algorithmic}[1]
\Require{$\rho \geq 1$}
\Ensure{$X_k$}
\State{$NumChanges \leftarrow 1$}
\State{ $R(v) \leftarrow P(v)$}
\While{$ NumChanges >0$}
	\For{all vertex $v\in V$, \emph{\textbf Do in Parallel} }
		\State{$ r= \min (R(v), R(P(v))) $}
		\If{$r \neq R(v) $}
			\State{$R(v)= v$}
			\State{$NumChanges =NumChanges+1$}
		\EndIf
		\If{$v = R(v) $}
			\State{\emph{\textbf Break the loop at $v$} }
			\State $P(v)=v$
		\EndIf
	\EndFor
\EndWhile
\State \emph{\textbf {Correction} }
\State $CC[v] = R(v)$

\end{algorithmic}\caption{}
\end{algorithm}

The Algm{[}z{]} works by when a vertex $v$ finds loop $v=\mathcal{A}(v)$,
then $v$ has the minimum vertex id in the sequence $\left\{ v,P(v),P^{2}(v),\ldots\right\} $.
Thus, $v$ must be root of the tree. Therefore, we reset the vertex
$v$, by setting its parent to itself $P(v)=v$. Also when the loop
detection converges, $\mathcal{A}(v)$ has the minimum vertex label
among all its ancestors. Thus, the correct value of $CC[v]$ should
be $\mathcal{A}(v)$. So at the end of loop detection step all vertex
$v$ will have aquired a valid value. 

\subsection{Fault-tolerant label propagation algorithm}

Since the self-stabilization step has cost $Vlog(V)$, it is prohibitively
expensive to be performed in every iteration. 

To ensure the self-stabilization property, previously{[}cite{]} we
performed the correction step at regular interval. However, in case
of label propagation algorithm, number of iteration is $\mathcal{O}(log(d))$,
where $d$ is the diameter of the graph. Thus, for many graph label
propagation algorithm conerges in 5 to 10 iterations. 

Due to these circumstanes, we only perform correction step when the
label propgation algorithm reports convergence. If the correction
step finds that the algorithm is in valid state, and algorithm has
converged than algorithm has converged. If the correction step reports
that algorithm is in an invalid state, then it tries to bring the
algorithm back to valid state and restarts the label propagation iteration.
However, this state is usually very close to the final state and it
only takes a small fraction of iteration to converge{[}ref to result{]}. 


\section{Empirical Results }
\label{sec:results}
%%% result.tex 
%% displays experimental setup, and description of matrices 
We performed a series of experiments to test robustness and overhead of our algorithm described in~\refsec{sec:connected}. 

%
\subsection{Fault Injection Methodology}
Since memory accesses are most performance critical part in~\sv computation, we inject faults in memory access pattern. There are two main memory accesses in~\sv iteration: traversing adjacency list for each vertex; and 
accessing $CC$ array for each vertex in adjacency list. 

In the first case, before each~\sv sweep, we randomly select $f|E|$ edges, where $f$ is fault-rate. Let $e=(v,u)$ be  one of the selected edges. When we encounter $e$ while traversing adjacency list of $v$, we flip one of the bits of $u$---the vertex which $e$ is pointing---randomly. 
Therefore, due to the fault, $v$ will visit flipped vertex $\hat{u}$, instead of accessing $u$. 

In the second case, we will again choose another set of $f|E|$ edges. Let $e=(v,u)$ be  one of the selected edges. When we encounter $e$ while traversing adjacency list of $v$, we flip one of the bits of $CC[u]$.
Therefore, due to the fault, $v$ will visit the correct vertex, however, it see incorrect value of  $CC[u]$.
It should be noted that $CC[u]$ will be accessed multiple times in a~\sv iteration, and we assume that 
all accesses to $CC[u]$ in an iteration are independent. In other words, if $CC[u]$ is accesses while visiting 
vertex $v_{1},\ v_{2}$, and if $CC[u]$ is corrupted while visiting $v_{1}$, then $CC[u]$ may or may not be corrupted when it is accessed while visiting $v_{2}$.


\subsection{Experimental Setup}



% type of matrices 
\paragraph{Test Graphs}
The graphs used in our tests are listed in~\reftab{tab:graphs}. 
These graphs are taken from the 10th Dimacs Implementation Challenge~\cite{Bader-dimacs-graph2014}, come from various real and synthetic applications. 
\begin{table*}[]
\centering
\caption{List of matrices used for experimentation}
\label{tab:matrix}
\begin{tabular}{llll}
Matrix Name              & Source                      & \#Vertices & \#Edges  \\
astro-ph                 & collaboration network       & 16706      & 242502   \\
audikw1                  & UF Sparse Matrix Collection & 943695     & 77651847 \\
caidaRouterLevel         & Clustering                  & 192244     & 1218132  \\
cnr-2000                 &                             & 325557     & 2738969  \\
coAuthorsDBLP            &                             & 299067     & 977676   \\
coPapersDBLP             &                             & 540486     & 15245729 \\
cond-mat-2005            &                             & 40421      & 175691   \\
delaunay\_n18            &                             & 262144     & 786396   \\
er-fact1.5-scale20       &                             & 1048576    & 10904496 \\
G\_n\_pin\_pout          &                             & 100000     & 501198   \\
kron\_g500-simple-logn18 &                             & 262144     & 10582686 \\
ldoor                    &                             & 952203     & 22785136 \\
preferentialAttachment   &                             & 100000     & 499985   \\
rgg\_n\_2\_18\_s0        &                             & 262144     & 1547283 
\end{tabular}
\end{table*}

\paragraph{Testbed}
\begin{table}[]
\centering
\caption{Testbeds used for performance evaluation.}
\label{tab:sys_info}
\begin{tabular}{ll}
Prop                 & SNB20c       \\
Sockets$\times$Cores & 2$\times$8   \\
Clock Rate           & 2.4GHz       \\
DRAM capacity        & 128GB        \\
DRAM Bandwidth       & 72GB/s      
\end{tabular}
\end{table}
We prototyped baseline and fault tolerant implementation using $C$ language. 
We used the Intel C Compiler (ICC 15.0.0), with highest level of
 optimization $-O3$ to compile our benchmarks.
We ran all our experiments on SNB16c, key properties of the systems are listed in~\reftab{tab:sys_info}


\subsection{Convergence in presence of faults}


\subsection{Overhead of fault detection and correction}
%% compare it with double modular redundancy 



\section{Related work}
\label{sec:related}
Here goes the related work \cite{shantharam2012fault}.

\section{Conclusion and Future Work}
\label{sec:conclusion} \lyxframe{Conclusion 
% and Future Work
 }
\begin{block}{Conclusion}
\begin{itemize}
\item We introduced the ideas of self-correcting algorithm to build fault tolerant algorithms. 
	
\item We presented a self-correcting label propagation algorithm for graph connected component problem.
\item Key steps involved:
	\begin{itemize}
	\item Analyze valid and invalid state;
	\item Use self-correction hypothesis to simplify invalid state detection;
	\item Use previous valid states to recover from invalid state. 
	\end{itemize}

	
\item Asymptotically lower overhead for fault detection and correction. 
\item 10-35\%  increases in execution time for one error for 64 memory operations.

	
\end{itemize}
\end{block}

% \begin{block}{Outlook}
% \begin{itemize}
% \item Comparing self-stabilization formulation with self-correcting formulation.
% \item Fill this up.
% \end{itemize}
% \end{block}


\lyxframeend{}



% \subsection*{Acknowledgments}

% \input{ack}

\bibliographystyle{ieeetr}
\bibliography{sao} 

\end{document}
