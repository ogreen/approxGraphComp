\documentclass[english]{sig-alternate-05-2015}
\usepackage[T1]{fontenc}
\usepackage[latin9]{inputenc}
\usepackage{verbatim}

%% without following commands, amsthm complains proof being already defined 
\let\proof\relax
\let\endproof\relax
\usepackage{amsthm}
\usepackage{amstext}
\usepackage{graphicx}

\usepackage{mdwlist} % compact lists

\usepackage{relsize} % relative font sizes
\usepackage{todonotes} % private notes

\usepackage{soul}

\usepackage{hyperref}
\hypersetup{bookmarks=true,bookmarksnumbered=true,pdfstartview=FitB,colorlinks=true,pdfborder=0 0 0}

\usepackage{amsmath}

\usepackage{xspace}

\usepackage[sort&compress]{natbib}

\makeatletter

%%%%%%%%%%%%%%%%%%%%%%%%%%%%%% LyX specific LaTeX commands.
%% Because html converters don't know tabularnewline
\providecommand{\tabularnewline}{\\}
%% A simple dot to overcome graphicx limitations
\newcommand{\lyxdot}{.}


%%%%%%%%%%%%%%%%%%%%%%%%%%%%%% Textclass specific LaTeX commands.
\theoremstyle{plain}
\newtheorem{thm}{\protect\theoremname}
\theoremstyle{definition}
\newtheorem{defn}{\protect\definitionname}

%%%%%%%%%%%%%%%%%%%%%%%%%%%%%% User specified LaTeX commands.
\usepackage{algorithm}


\usepackage{algpseudocode}

\@ifundefined{showcaptionsetup}{}{%
 \PassOptionsToPackage{caption=false}{subfig}}
\usepackage{subfig}
\makeatother

\usepackage{babel}
\providecommand{\definitionname}{Definition}
\providecommand{\theoremname}{Theorem}

\DeclareCaptionType{copyrightbox}
%========== Custom typographic commands ==========
\newcommand{\emailAddr}[1]{\texttt{\smaller {#1}}}
\newcommand{\mailtoLink}[2]{\href{mailto:#2}{\emailAddr{#1}}}
\newcommand{\refSection}[1]{Section~\ref{#1}}
\newcommand{\refsec}[1]{\S~\ref{#1}}
\newcommand{\refequ}[1]{eq.~\eqref{#1}}
\newcommand{\refEquation}[1]{Equation~\eqref{#1}}
\newcommand{\refFigure}[1]{Figure~\ref{#1}}
\newcommand{\reffig}[1]{fig.~\ref{#1}}
\newcommand{\refDefinition}[1]{Definition~\ref{#1}}
\newcommand{\refdef}[1]{defn.~\ref{#1}}
\newcommand{\refAlgorithm}[1]{Algorithm~\ref{#1}}
\newcommand{\refalg}[1]{alg.~\ref{#1}}
\newcommand{\refTable}[1]{Table~\ref{#1}}
\newcommand{\reftab}[1]{table~\ref{#1}}
\newcommand{\refTheorem}[1]{Theorem~\ref{#1}}
\newcommand{\refthm}[1]{thm.~\ref{#1}}

%========== Document Specific commands ==========


\newcommand{\sv}{SV}
\newcommand{\graphname}[1]{\emph{#1}}




\begin{document}

\setcopyright{acmcopyright}

\title{A Fault Tolerant Connected Component Algorithm }

\numberofauthors{1} 
  \author{
  \alignauthor
    Piyush Sao, Oded Green, Chirag Jain, Richard Vuduc \\
      \affaddr{School of Computational Science and Engineering}\\
      \affaddr{Georgia Institute of Technology}\\
      \email{\{piyush3,ogreen3,cjain3,richie\}@gatech.edu}
  %
  % \alignauthor
  %   Xiaoye S. Li\\
  %     \affaddr{Computational Research Division }\\
  %     \affaddr{Lawrence Berkeley National Laboratory}\\
  %     \email{xsli@lbl.gov}
  }


\maketitle

\begin{abstract}
% \todo{redo}
This paper presents a fault tolerant label propagation based graph connect component algorithm, that can converge to correct solution even in presence soft transient faults. 

%
First, we develop a set of conditions on state variables of the algorithm that ensures
convergence to correct solution. We augment the algorithm 
with additional state variable to make verification of these conditions becomes computationally tractable. 
% valid state not defined
 Finally, we add a local correction scheme which ensures 
the algorithm comes to a valid state after a faulty execution. 

%
We present experimental results on wide range of matrices to show that the new algorithm can withstand high fault rates, with minimal storage and computational overhead. Additionally, we present analytical models for overhead of fault detection and correction. 

\end{abstract}


\keywords{Fault-tolerance, Graph-algorithm, Connected Component, Bit-flip}


\section{Introduction}


The exceeding growth of big datasets has created a need for using distributed systems and 
accelerators for real-time analytics, including for graph and social network analytics. Social 
networks such Facebook and Twitter are constantly growing  at time of writing have over #### 
members with **** relationships between these members.
The decrease of transistor sizes and the improved power consumption, as part of Moore's laws, has 
brought a new challenge with it - an increased number of faults due to random bit flipping. This 
problem is very likely to become even more challenging as transistor sizes continue to decrease and 
the power envelope restrictions become even tighter. As larger systems, with a higher thread count, 
become ubiquitous so will the problem of power related faults.

In this work, we tackle the problem of faults in a connected component algorithm that is 
propagation based. We modify the parallel algorithm of Shiloach-Vishkin \cite{shiloachvishkin} and 
make it fault-tolerant by augmenting the data structures used by the algorithm and by adding a 
detection and correction scheme at the end of every iteration. By detecting and correcting the 
error at the end of each iteration we are able to limit the propagation of the error to the 
remainder of the graph in the following iterations.

Main contributions:
* Little overhead when there aren't any faults.
* Same number of iterations as a fault-free execution for reasonable error rates.
* Overhead if rather constant in time until the number of threads is significantly high.
* 




\label{sec:intro}

\section{Fault Model}

We term fault as any instance of hardware deviating from its expected
behavior. Impact of such faults on the application can vary  depending on the
location of the fault. Based on the impact of the faults, they can be
classified into two broad categories: hard fault and soft faults. A hard fault
causes the application to terminate prematurely. For instance, failing of nodes and network fall
into this category. On the other hand, soft faults may not cause the
application to abort. Even so, a soft fault can lead to an incorrect result,
which we term as the failure. Bitflips in memory and latches are examples of
soft faults. Interested readers can find a more detailed discussion on faults
elsewhere  \cite{hoemmen2011ftgmres}.

In this paper, we only deal with soft faults. A particularly insidious
manifestation of soft-faults is silent data corruption. Silent data corruption
occurs when a soft fault leads to corruption of entire intermediate variables,
without notifying the application. In such cases, the application may arrive
at an incorrect solution and yet, report it as correct solution. Thus, silent
data corruption can lead to serious reliability issues in the computing. 


The importance of having algorithmic level fault tolerance has already been
explored for a number of numerical computations (see \cref{sec:related}). Graph
computations are equally susceptible to soft faults as numerical computation.
Researchers have started looking at resilient discrete computation only
recently (\TODO{cite})


\label{sec:fault}

\section{Fault tolerant Label propagation algorithm for graph connected component}

\paragraph{Label Propagation algorithm in Fault Free execution}

\begin{figure*}

\includegraphics[width=0.2\paperwidth]{figure/sv1}\includegraphics[width=0.2\paperwidth]{figure/sv2}\includegraphics[width=0.2\paperwidth]{figure/sv3}\includegraphics[width=0.2\paperwidth]{figure/sv4}\includegraphics[width=0.2\paperwidth]{figure/sv5}\caption{These sub-figures conceptually show how the connected component id
propagates through the graph as time evolves. Subfigures represent
snapshots of the algorithm at different times. For simplicity, this
example assumes that all the shown vertices are connected. Initially
the number of connected components is equal to the number vertices.
(a) Depicts the initial state in which each vertex is in its own component.
(b)-(d) depict that some vertices belong to the same connected component
yet may require multiple label updates (in either the same iteration
or a separate iteration). (e) is the final state in which there is
a single connected component.}

\end{figure*}

\begin{table}

\centering
\footnotesize
\caption{Symbols used in the fault free \sv algorithm and in the new fault tolerant algorithm.}
\ra{1.2}
\begin{tabular}[t]{@{}lll@{}}
\toprule
Symbol 					& Decription 						& Size\\
\midrule
$V$ 					& Vertices in the graph 			& O(V)\\
$E$ 					& Edges in the graph 				& O(E)\\
$adj(v)$ 				& Adjacency list for vertex $v\in V$& \\
\CCVAL 					& Connected component array 		& O(V)\\
\CCVAL$^i$ 				& \specialcell{Connected component array \\after iteration} $i$ & O(V)\\
\CCVAL$^\infty$ 		& \specialcell{The final connected component \\mapping upon algorithm completion}. & O(V)\\

						& Fault free \\

$H$ 					&  & O(V)\\
\bottomrule 
\end{tabular}
\label{tab:symbols}

\end{table}

The aim of label propagation algorithm is to mark every vertex in
a given connected component with a pre-defined common label. This
common label is same within each connected component, and different
across different connected component. There are many choice for the
common label. As a convention, we consider minimum vertex-id in the
connected component as the final common label. 

The label propgation algorithm keeps an array $CC$ of current labels
for all vertices. For each vertex $v$, its label $CC[v]$ is initalized
with its vertex-id $CC[v]=v$. In every iteration, each vertex updates
its label by calculating minimum label of all its neighbours and itself:
\begin{equation}
CC^{i+1}[v]=\min_{u\in\mathcal{N}(v)}CC^{i}[u],\label{eq:lp_update_eqn}
\end{equation}

where $\mathcal{N}(v)=\left\{ v,adj(v)\right\} $ is the defined as
immidiate neighbourhood of $v$. Thus, through out the iteration minimum
vertex id propagates to all the vertces in the connect component.
The iteration converges when there are no more label changes in the
graph. 

Depending on the constrints, equation \ref{eq:lp_update_eqn} can
be implmented in two ways. In the first way, we use two different
arrays to store $CC^{i+1}$ and $CC^{i}$. We refer to this implementation
is Sync LP algorithm. In an another way, we overwrite $CC^{i+1}$
on $CC^{i}$. We refer to this version as Async LP algorithm. Depending
on architecture and programming model, the two variants may have different
performance characteristics. In subsequent discussion, we assume Async
LP algorithm. Yet, our results are equally applicable to both instance
of the LP algorithm.

Each iteration of LP, visits all the vertex and edges once and thus
costs $\mathcal{O}(V+E)$. The LP algorithm requires $\mathcal{O}(d)$
iterations to converge, where $d$ is the diameter of the graph. We
may use short-cutting to bound the number of iteration to $\mathcal{O}(log(d))$
{[}insert citation{]}. However, in practice async LP algorithm only
takes a few more iteration than implementing full short cutting step,
without the cost of short cutting. 

\label{sec:connected}

\section{Results}
%%% result.tex 
%% displays experimental setup, and description of matrices 
We performed a series of experiments to test robustness and overhead of our algorithm described in~\refsec{sec:connected}. 

%
\subsection{Fault Injection Methodology}
Since memory accesses are most performance critical part in~\sv computation, we inject faults in memory access pattern. There are two main memory accesses in~\sv iteration: traversing adjacency list for each vertex; and 
accessing $CC$ array for each vertex in adjacency list. 

In the first case, before each~\sv sweep, we randomly select $f|E|$ edges, where $f$ is fault-rate. Let $e=(v,u)$ be  one of the selected edges. When we encounter $e$ while traversing adjacency list of $v$, we flip one of the bits of $u$---the vertex which $e$ is pointing---randomly. 
Therefore, due to the fault, $v$ will visit flipped vertex $\hat{u}$, instead of accessing $u$. 

In the second case, we will again choose another set of $f|E|$ edges. Let $e=(v,u)$ be  one of the selected edges. When we encounter $e$ while traversing adjacency list of $v$, we flip one of the bits of $CC[u]$.
Therefore, due to the fault, $v$ will visit the correct vertex, however, it see incorrect value of  $CC[u]$.
It should be noted that $CC[u]$ will be accessed multiple times in a~\sv iteration, and we assume that 
all accesses to $CC[u]$ in an iteration are independent. In other words, if $CC[u]$ is accesses while visiting 
vertex $v_{1},\ v_{2}$, and if $CC[u]$ is corrupted while visiting $v_{1}$, then $CC[u]$ may or may not be corrupted when it is accessed while visiting $v_{2}$.


\subsection{Experimental Setup}



% type of matrices 
\paragraph{Test Graphs}
The graphs used in our tests are listed in~\reftab{tab:graphs}. 
These graphs are taken from the 10th Dimacs Implementation Challenge~\cite{Bader-dimacs-graph2014}, come from various real and synthetic applications. 
\begin{table*}[]
\centering
\caption{List of matrices used for experimentation}
\label{tab:matrix}
\begin{tabular}{llll}
Matrix Name              & Source                      & \#Vertices & \#Edges  \\
astro-ph                 & collaboration network       & 16706      & 242502   \\
audikw1                  & UF Sparse Matrix Collection & 943695     & 77651847 \\
caidaRouterLevel         & Clustering                  & 192244     & 1218132  \\
cnr-2000                 &                             & 325557     & 2738969  \\
coAuthorsDBLP            &                             & 299067     & 977676   \\
coPapersDBLP             &                             & 540486     & 15245729 \\
cond-mat-2005            &                             & 40421      & 175691   \\
delaunay\_n18            &                             & 262144     & 786396   \\
er-fact1.5-scale20       &                             & 1048576    & 10904496 \\
G\_n\_pin\_pout          &                             & 100000     & 501198   \\
kron\_g500-simple-logn18 &                             & 262144     & 10582686 \\
ldoor                    &                             & 952203     & 22785136 \\
preferentialAttachment   &                             & 100000     & 499985   \\
rgg\_n\_2\_18\_s0        &                             & 262144     & 1547283 
\end{tabular}
\end{table*}

\paragraph{Testbed}
\begin{table}[]
\centering
\caption{Testbeds used for performance evaluation.}
\label{tab:sys_info}
\begin{tabular}{ll}
Prop                 & SNB20c       \\
Sockets$\times$Cores & 2$\times$8   \\
Clock Rate           & 2.4GHz       \\
DRAM capacity        & 128GB        \\
DRAM Bandwidth       & 72GB/s      
\end{tabular}
\end{table}
We prototyped baseline and fault tolerant implementation using $C$ language. 
We used the Intel C Compiler (ICC 15.0.0), with highest level of
 optimization $-O3$ to compile our benchmarks.
We ran all our experiments on SNB16c, key properties of the systems are listed in~\reftab{tab:sys_info}


\subsection{Convergence in presence of faults}


\subsection{Overhead of fault detection and correction}
%% compare it with double modular redundancy 


\label{sec:results}

\section{Related work}
Here goes the related work \cite{shantharam2012fault}.
\label{sec:related}

\section{Conclusion and future work}
\lyxframe{Conclusion 
% and Future Work
 }
\begin{block}{Conclusion}
\begin{itemize}
\item We introduced the ideas of self-correcting algorithm to build fault tolerant algorithms. 
	
\item We presented a self-correcting label propagation algorithm for graph connected component problem.
\item Key steps involved:
	\begin{itemize}
	\item Analyze valid and invalid state;
	\item Use self-correction hypothesis to simplify invalid state detection;
	\item Use previous valid states to recover from invalid state. 
	\end{itemize}

	
\item Asymptotically lower overhead for fault detection and correction. 
\item 10-35\%  increases in execution time for one error for 64 memory operations.

	
\end{itemize}
\end{block}

% \begin{block}{Outlook}
% \begin{itemize}
% \item Comparing self-stabilization formulation with self-correcting formulation.
% \item Fill this up.
% \end{itemize}
% \end{block}


\lyxframeend{}


\label{sec:conclusion}

\section{Acknowledgment}
\input{ack}

% \balance
\bibliographystyle{abbrv}
% \addbibresource{sao.bib}
\bibliography{sao}


\end{document}
