% \begin{itemize}
% \item %
% \begin{comment}
% \begin{itemize}
% \item difficulty in checking above condition for validity 
% \begin{itemize}
% \item monotonicity and root can be checked 
% \item checking tree 
% \begin{itemize}
% \item is equivalent to loop detection 
% \item O(V) algorithm exist, not vertices centrinc, not scalable 
% \item need new algorithm 
% \end{itemize}
% \end{itemize}
% \end{itemize}
% \end{comment}
% \item %
% \begin{comment}
% \begin{itemize}
% \item Loop detection in graph G=(V, \{V,P(V)\}) 
% \begin{itemize}
% \item idea of the proof: if it is valid tree then $min(P(v),p^{2}(v)..)$
% should be less then $v$ 
% \item concrete theorem and proof 
% \item How to implement it in vertex centric model 
% \item algorithm 
% \end{itemize}
% \end{itemize}
% \end{comment}
% \item %
% \begin{comment}
% \begin{itemize}
% \item Complete correction step 
% \begin{itemize}
% \item elements of correction step: 
% \begin{itemize}
% \item conditions to check 
% \item order of checking 
% \item correction step 
% \end{itemize}
% \item frequncy of correction step 
% \item item correction step algorithm 
% \item self-stabilzing algorithm 
% \end{itemize}
% \end{itemize}
% \end{comment}
% \item %
% \begin{comment}
% Cost of correction
% \end{comment}
% \end{itemize}
In this section, we will describe the design of the self-stabilizing
label propagation algorithm. We start with decribing the state of
the algorithm and a condition for a valid state. However, the condition
for valid state is not easy to verify computationally. To overcome
this issue, in we augmented the data structure with redundancy to
allow state verification. We use the augmented state as previously,
and present new conditions for valid states. And finally we give efficient
algorithm to verify these algorithms in cost effective manner.

% \begin{comment}
% \textbf{\emph{I}}\emph{ntro to terminology, states and effect of faults} 
% \begin{itemize}
% \item states of LP algorithm 
% \item effect of fault on valid state 
% \item valid state condition, 
% \end{itemize}
% \end{comment}


\subsubsection*{State of LP algorithm}

In algo{[}REF{]}, the current connected component label array $CC$
describes the state of LP algorithm. We use $CC^{i}[v]$ to denote
the current label for vertex $v$ in $i$-th iteration. We denote
the final label of vertex $v$ in fault-free execution as $CC^{\infty}[v]$. 

\subsubsection*{Effect of fault in LP state}

\subsubsection*{Valid states}
\begin{thm}
A connected component array $CC$ is a valid state-{}-{}-i.e., a fault-free
execution of algorithm (CITE) starting from $CC$ will converge to
the correct solution-{}-{}-if, for all vertices $v$, $CC^{\infty}[v]\leq CC[v]\leq v$.
\end{thm}

\subsection{\emph{Self-correcting Label Propagation Algorithm}}

We described self-correcting label propagation algorithm for synchronoous
label propagation algorithm. Self-correcting algorithm corrects the
current state using previous valid state. 

To do so, we show that if $CC^{i-1}$ is valid state then $CC^{i}$
is valid if 
\begin{equation}
CC^{i}[v]=CC^{i-1}[u]\ where\ u\in\mathcal{N}(v)\ \forall v
\end{equation}

However, verifying eq{[}x{]} still requires $O(V+E)$ work, thus inefficient.
However, self-correcting algorithm uses an auxilliary data structure
called parent array. Parent of vertex $v$, denoted by $P[v]$ , refers
to vertex that 
% \begin{itemize}
% \item introduce parent array 
% \item what do you check and how do you check 
% \item Limitations 
% \begin{itemize}
% \item fault propagation 
% \item is not applicable to asynchronous version 
% \end{itemize}
% \end{itemize}

\subsection{\emph{Self-stabilzing Label Propagation Algorithm} }

The main difference between self-correcting label propagation algorithm
and self-stabilizing label propagation algorithm is that self-stabilizing
label propagation algorithm doesn't require a previous valid state
$S^{i-1}$ to bring the algorithm to a valid state. We can formally
pose this question as follows.

\paragraph*{Problem Statement:}

Given a graph $G=\left(V,E\right)$ and an arbitrary state for label
propagation $S=\left(CC,P\right)$, determine if $S$ is a valid state.
If $S$ is invalid, then construct a state $S^{*}$such that $S^{*}\approx S$,
and $S^{*}$ is a valid state. 

\subsubsection{Properties of LP state in fault free execution}

We describe two key properties of a state $S$ in fault free execution
and later we prove that for any state that satisfy these properties
is a valid state. 
\begin{enumerate}
\item $CC[v]\leq v$;
\item $P(v)\in\mathcal{N}(v)$;
\item $CC[P(v)]\leq CC[v]$; and 
\item Directed graph described by parent array $H=(V,E_{H})$, where $E_{H}=\left\{ (v,P(v)),\forall v\in V\right\} $
describes a forest. 
\end{enumerate}
Self-correcting label propagation algorithm uses property 1 and 2,
and additionally assumes that $CC^{i-1}$ is a valid state. 

Property 3 describes the range of valid values of $CC[v]$. In a synchronous
label propagation algorithm, we have $CC^{i}[v]=CC^{i-1}[P(v)]$,
and since label decreases over the iteration, thus latest $CC^{i}[P(v)]\leq CC^{i-1}[P(v)]$.
Therefore, $CC^{i}[P(v)]\leq CC^{i}[v]$. While we do not show here
in detail, this relation holds good in fault free execution of asynchronous
case also. 

Property 4 describes the structure of parent array. Make a gifure
and explain. 

\subsubsection{Sufficient condtions for a valid LP state }

We show that the former four conditions are sufficient for a valid
state. 
\begin{thm}
A label propagation state $S=\left\{ CC,P\right\} $ is a valid state
if for all vertices $v$, we have following conditions met:
\end{thm}
\begin{enumerate}
\item $CC[v]\leq v$;
\item $P(v)\in\mathcal{N}(v)$;
\item $CC[P(v)]\leq CC[v]$; 
\item If $P(v)=v$, then $CC[v]=v$; and 
\item Directed graph described by parent array $H=(V,E_{H})$, where $E_{H}=\left\{ (v,P(v)),\forall v\in V\right\} $
describes a forest.
\end{enumerate}
\begin{IEEEproof}
If $H=\left(V,E_{H}\right)$ is a forest then, for any vertex $v$,
the sequence $\left\{ P(v),P^{2}(v),\ldots\right\} $ convereges to
the root of the tree in the forest $H$. Let's denote the convergent
of the sequence as $P^{\infty}(v)$. 

Let $u=P^{\infty}(v$). Since $P(u)=P(P^{\infty}(v))=P^{\infty}(v)=u$,
therefore from condition 4:
\begin{equation}
CC[u]=u
\end{equation}

If condition-2 holds for all the vertices $\left\{ v,P(v),P^{2}(v),\ldots\right\} $,
then $v$ and $P^{\infty}(v)=u$ are in same connected component.
Therefore,
\begin{equation}
CC^{\infty}[v]=CC^{\infty}[u].
\end{equation}

Since label of $u,$will monotonically decrease in susequent iteration,
therefore 
\begin{equation}
u\ge CC^{\infty}[u]
\end{equation}

From condition-3 $CC[v]\geq CC[P(v)]\geq CC[P^{2}(v)]...$
\begin{equation}
CC[v]\geq CC[P^{\infty}(v)]=CC[u]=u
\end{equation}

Combining eq(x) and eq(y) and eq(z), we get $CC[v]\ge u\geq CC^{\infty}[u]=CC^{\infty}[v]$
\begin{equation}
CC[v]\geq CC^{\infty}[v]
\end{equation}

Combining eq(x) with condition 1, we get 
\[
CC^{\infty}[v]\le CC[v]\le v
\]

Eq(x) holds for all the vertices $v\in V$, therefore by Thm(1), $S=\left\{ CC,P\right\} $
is a valid state.
\end{IEEEproof}
%
Note that Thm(2) describes a set of sufficient condition. In general,
there can be a state from which algorithm will converge to correct
solution yet not satisfy Thm(2), leading to false positives. However,
false-positives will only cause a small amount of additional computation,
which is not a big concern in this case. 

\subsection{Verifying state validity}

All the conditions of Thm(2) except the last one, can be verified
locally for any vertex in $\mathcal{O}(1)$ operations. To verify
$H=(V,E_{H})$ is a forest, is equivalent to verifying that $H$ does
not have any loops, which is not a local operation.

\subsubsection{Loop detection in $H$}

In sequential case, we can use Tarzen's strongly connected component
algorithm to find the loops in the graph $H$. Tarzen's strongly connected
component algorithm requires $\mathcal{O}(V+E_{H})$ operations. But
since, $|E_{H}|=|V|$, therefore detecting loop in $H$ requires $\mathcal{O}(V)$
operations in total, which is same as cost of other operations in
sequential case.

In parallel case, Tarzen's strongly connected component algorithm
(or any other algorithm based on BFS or DFS) are not suitable as:
(a) they are inhrently sequential and have a depth of $\mathcal{O}(V)$;
and (b) they can not be efficiently expressed in vertex centric programming
model as label propagation algorithm. Therefore, we need a new loop
detection algorithm which is parallel and can be expressed in vertex
centric programming model. 

\subsubsection{Parallel Loop detection in $H$}

\begin{algorithm}
\begin{algorithmic}[1]
\Require{$\rho \geq 1$}
\Ensure{$X_k$}
\State{$NumChanges \leftarrow 1$}
\State{ $R(v) \leftarrow P(v)$}
\While{$ NumChanges >0$}
	\For{all vertex $v\in V$, \emph{\textbf Do in Parallel} }
		\State{$ r= \min (R(v), R(P(v))) $}
		\If{$r \neq R(v) $}
			\State{$R(v)= v$}
			\State{$NumChanges =NumChanges+1$}
		\EndIf
		\If{$v = R(v) $}
			\State{\emph{\textbf Report Loop} }
		\EndIf
	\EndFor
\EndWhile
\end{algorithmic}\caption{}
\end{algorithm}

Our parallel loop detection is based on the idea that if $v$ is a
part of the loop in $H$, then $v$ will re-appear in the sequence
$\left\{ P(v),P^{2}(v),\ldots\right\} $. Let's call the minimum element
in the sequence $\mathcal{A}(v)$:
\[
\mathcal{A}(v)=\min\left\{ P(v),P^{2}(v),\ldots\right\} 
\]

If there is no loop in the, then $v$ will not appear in the sequence
$\left\{ P(v),P^{2}(v),\ldots\right\} $, thus $v$ can not be equal
to $\mathcal{A}(v)$. Therefore, $v=\mathcal{A}(v)$ is only possible
if there is a loop in $H$ which consist of vertices $\left\{ v,P(v),P^{2}(v),\ldots\right\} $.
Thus to detect any loop in $H$, we calculate $\mathcal{A}(v)$ for
all the vertices and compare it with $v$. 

We calculate $\mathcal{A}(v)$ in parallel using pointer jumping technique.
The Algm{[}x{]} describes the pseudocode for parallel loop detection
in $H$. ADD more text here

\paragraph*{Cost of loop detection}

If there is no loop in $H$, then Algm{[}x{]} requires $\lceil log_{2}(h)\rceil+1$
iterations to converge, where $h$ is the maximum height among all
the trees in the forest $H$. If there are cycles in $H$, and if
length of maximum cycle is $c$then, Algn{[}x{]} takes $\lceil log_{2}(c)\rceil+2$
iterations to detect it. So it need a total of $\max\left(\lceil log_{2}(h)\rceil+1,\lceil log_{2}(c)\rceil+2\right)$
iterations. In each iteration, we perform $\mathcal{O}(V)$ work.
Thus total cost of loop detection is $\mathcal{O}(Vlog(V))$ as $c,\ h\leq|V|$. 

\subsection{Correction step}

Correction step has two parts: detection of invalid component of states;
and bring them to a correct state. Using Algm{[}x{]} and Thm{[}y{]},
we design the correction step as follows. First we ensure that graph
$H$ is a proper subgraph of the input graph $G$, by verifying $P(v)\in\mathcal{N}(v)$
for all vertex $v$. If for some vertex $P(v)\notin\mathcal{N}(v)$,
then we reset vertex $v$ by setting $CC[v]=v$ and $P(v)=v$. After
this we check, if a vertex $v$ is a root of any tree, then $CC[v]=v$.
It should be noted that in both check, if we find that state of $v$
is invalid, it also means that state of all the descendent of $v$
is also invalid. However, all the otehr vertices which have invalid
state have not been informed yet. Once these checks and local correction
are performed, we use a modified algorithm to check for loops and
correct labels. 

\begin{algorithm}
\begin{algorithmic}[1]
\Require{$\rho \geq 1$}
\Ensure{$X_k$}
\State{$NumChanges \leftarrow 1$}
\State{ $R(v) \leftarrow P(v)$}
\While{$ NumChanges >0$}
	\For{all vertex $v\in V$, \emph{\textbf Do in Parallel} }
		\State{$ r= \min (R(v), R(P(v))) $}
		\If{$r \neq R(v) $}
			\State{$R(v)= v$}
			\State{$NumChanges =NumChanges+1$}
		\EndIf
		\If{$v = R(v) $}
			\State{\emph{\textbf Break the loop at $v$} }
			\State $P(v)=v$
		\EndIf
	\EndFor
\EndWhile
\State \emph{\textbf {Correction} }
\State $CC[v] = R(v)$

\end{algorithmic}\caption{}
\end{algorithm}

The Algm{[}z{]} works by when a vertex $v$ finds loop $v=\mathcal{A}(v)$,
then $v$ has the minimum vertex id in the sequence $\left\{ v,P(v),P^{2}(v),\ldots\right\} $.
Thus, $v$ must be root of the tree. Therefore, we reset the vertex
$v$, by setting its parent to itself $P(v)=v$. Also when the loop
detection converges, $\mathcal{A}(v)$ has the minimum vertex label
among all its ancestors. Thus, the correct value of $CC[v]$ should
be $\mathcal{A}(v)$. So at the end of loop detection step all vertex
$v$ will have aquired a valid value. 

\subsection{Fault-tolerant label propagation algorithm}

Since the self-stabilization step has cost $Vlog(V)$, it is prohibitively
expensive to be performed in every iteration. 

To ensure the self-stabilization property, previously{[}cite{]} we
performed the correction step at regular interval. However, in case
of label propagation algorithm, number of iteration is $\mathcal{O}(log(d))$,
where $d$ is the diameter of the graph. Thus, for many graph label
propagation algorithm conerges in 5 to 10 iterations. 

Due to these circumstanes, we only perform correction step when the
label propgation algorithm reports convergence. If the correction
step finds that the algorithm is in valid state, and algorithm has
converged than algorithm has converged. If the correction step reports
that algorithm is in an invalid state, then it tries to bring the
algorithm back to valid state and restarts the label propagation iteration.
However, this state is usually very close to the final state and it
only takes a small fraction of iteration to converge{[}ref to result{]}. 
